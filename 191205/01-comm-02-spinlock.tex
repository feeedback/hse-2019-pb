\subsection{Простая реализация}

\begin{frame}{Зачем}
	Довольно часто потоки не совсем независимы, а хотят взаимодействовать между собой.

	Классическая задача:
	\begin{itemize}
		\item Есть очередь задач.
		\item Один поток генерирует данные (producer) и добавляет их в очередь.
		\item Второй поток должен брать добавленные данные по очереди (consumer) и что-то с ними делать.
	\end{itemize}
	Например:
	\begin{itemize}
		\item Первый поток ждёт ввода с клавиатуры и кладёт считанные данные в буфер.
		\item Второй поток выполняет введённые команды (которые могут занять долгое время).
		\item Мы хотим уметь вводить команды, даже если предыдущая ещё выполняется.
	\end{itemize}
\end{frame}

\begin{frame}{Producer-Consumer}
	\svgimg{prod-cons}
\end{frame}

\begin{frame}[fragile]{Потокобезопасная очередь}
\begin{minted}{cpp}
class ThreadsafeQueue {
    ThreadsafeQueue() { pthread_mutex_init(&m, NULL); }
    ~ThreadsafeQueue() { pthread_mutex_destroy(&m); }
    void push(int x) {
        pthread_mutex_lock(&m);
        q.push(x);
        pthread_mutex_unlock(&m);
    }
    int pop() { ... }
    bool empty() { ... }
private:
    pthread_mutex_t m;
    queue<int> q;
};
\end{minted}
\end{frame}

\begin{frame}[fragile]{Первая попытка}
	Producer:
\begin{minted}{cpp}
while (true) {
    int data = read_int();
    q.push(data);
}
\end{minted}
	Consumer:
	\pause
\begin{minted}{cpp}
while (true) {
    while (q.empty()) {
    }
    process_data(q.pop());
}
\end{minted}
\end{frame}

\begin{frame}{Проблемы}
	\begin{itemize}
		\item Consumer активно ждёт событие от первого и тратит процессорное время.
		\item Даже если ничего не происходит, программа потребляет 100\% CPU.
		\item Consumer постоянно берёт и отпускает mutex, мешая producer'у.
		\item
			А если добавить задержку в consumer (проверять только каждые $X$ мс),
			то сильно увеличится задержка в обработке.
	\end{itemize}
\end{frame}

\begin{frame}{Есть гонка}
	\svgimg{queue-pop-race}
	Без новых \textit{примитивов синхронизации} не обойтись.
\end{frame}
