\section{Работа с состоянием}
\begin{frame}
	\tableofcontents[currentsection,currentsubsection]
\end{frame}

\begin{frame}[t,fragile]{Общая идея}
	Функция получает на вход состояние, а возвращает новое.

\begin{minted}{haskell}
insertWith :: Ord k =>
              (a -> a -> a) -> k -> a ->
              Map k a -> Map k a
\end{minted}

	Обычно состояние делают последним аргументом, тогда хорошо
	работает частичное применение и композиция:
\begin{minted}{Haskell}
insertWith (+) "foo" 1 >>> insertWith (+) "bar" 2
\end{minted}

	Демо!
\end{frame}

\begin{frame}[t,fragile]{Возврат значения и изменение состояния}
	Хотим <<откусить>> из начала строки число:
\begin{minted}{haskell}
import Data.Char

readInt :: String -> (Int, String)
readInt s =
    let (h, t) = span isDigit s in
    (read h, t)
\end{minted}

	Можно обобщить:
\begin{minted}{haskell}
type StateString a = String -> (a, String)
readInt :: StateString Int
readFloat :: StateString Float
\end{minted}
	Дальше можно навернуть ещё классов типов (вплоть до Monad)
	и получить синтаксический сахар (do-нотация)
\end{frame}

\begin{frame}[t,fragile]{Общая идея}
\begin{minted}{haskell}
getLine :: RealWorld -> (String, RealWorld)
data IO a = RealWorld -> (a, RealWorld)

getLine :: IO String
\end{minted}
	\t{RealWorld} никогда не используется в отрыве от \t{IO}.
	Вытащить значение из \t{IO} нельзя, иначе потеряется связь с миром.

	Но есть комбинаторы:
\begin{minted}{haskell}
return :: a -> IO a
(>>=) :: IO a -> (a -> IO b) -> IO b
\end{minted}
\end{frame}
