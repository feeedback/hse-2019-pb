\section{Практические грабли}
\subsection{Простое приложение на pthread}

\begin{frame}
	\tableofcontents[currentsection,currentsubsection]
\end{frame}

\begin{frame}{Что такое pthread}
	\begin{itemize}
		\item Стандартный интерфейс функций для работы с потоками (POSIX Threads).
		\item Есть реализации под Windows, Linux и другие ОС.
		\item Стандарт при разработке программ на C.
		\item Имена функций и типов начинаются с \t{pthread\_}.
		\item В Linux можно получить справку, набрав \t{man <имя функции>} в консоли.
		\item Под остальными "--- то же самое, но в гугле.
	\end{itemize}
\end{frame}

\begin{frame}[fragile]{Один поток}
	\svgimg{simple-one-thread}
\end{frame}

\begin{frame}[fragile]{Много потоков}
	\svgimg{simple-two-threads}
\end{frame}

\begin{frame}[fragile]{Пример кода}
	Демо \texttt{01-simple.cpp}
\end{frame}

\begin{frame}{Как живут потоки}
	\begin{itemize}
		\item При создании потока при помощи \t{pthread\_create} указывается функция и её аргумент "--- один указатель на что угодно.
		\item Вернуть функция тоже может указатель на что угодно.
		\item Поток завершается, когда функция делает \t{return} или \t{pthread\_exit}.
		\item Указатель на поток хранится в переменной типа \t{pthread\_t}.
		\item При создании потока он сразу начинает выполняться.
		\item \t{pthread\_join} делает следующее:
			\begin{enumerate}
				\item Ждёт окончания работы потока.
				\item Освобождает все ресурсы потока (стек).
				\item Возвращает то, что вернула функция потока.
			\end{enumerate}
		\item Когда \t{main} делает \t{return 0} или вы вызываете \t{exit(0)}, умирает весь процесс со всеми потоками.
		\item Но в \t{main} можно сделать \t{pthread\_exit}, если очень хочется, тогда процесс не умрёт, пока живы потоки.
	\end{itemize}
\end{frame}

\begin{frame}{Демо detached}
	Демо \texttt{02-detached.cpp}
	% data стала глобальной, arg не нужен
	% pthread_detach (падает)
	% убираем pthread_join (не всегда успевает отработать)
	% pthread_exit в main (теперь окей)
\end{frame}

\begin{frame}[t]{Несколько замечаний про pthread}
	Про потоки и pthread:
	\begin{itemize}
		\item Использовать \t{void* arg} и возвращаемое значение для передачи данных необязательно.
		\item Вся память внутри процесса одинаково доступна всем потокам на чтение и запись.
		\item \t{void* arg} возникает только тогда, когда надо запустить потоки на разных данных.
		\item
			Что произойдёт, если мы забудем \t{join} и \t{main} завершится до начала \t{worker}?
			Предполагаем, что процесс при этом не умрёт.
			\only<2->{Неопределённое поведение "--- \t{worker} попытается изменить переменную \t{data}, которая уже исчезла.}
	\end{itemize}
\end{frame}

\begin{frame}{Кто освобождает ресурсы?}
	На самом деле в pthread есть два типа потоков: joinable и detached.

	Joinable:
	\begin{itemize}
		\item Тип по умолчанию.
		\item На таком потоке должен быть ровно один раз вызыван метод \t{pthread\_join}, который освободит ресурсы и сообщит, что поток вернул.
		\item Если не вызвать "--- ресурсы не будут освобождены до конца программы.
		\item Если вызвать дважды "--- второй вызов может уронить программу или вернуть неверный результат.
	\end{itemize}

	Detached:
	\begin{itemize}
		\item Система автоматически освободит ресурсы как только поток завершится.
		\item Нельзя вызывать \t{pthread\_join} и получать возвращаемое значение "--- его негде хранить после окончания работы.
	\end{itemize}
\end{frame}

\begin{frame}{В других системах}
	\begin{itemize}
		\item Joinable/detached также используется в Java.
		\item В Windows (не в pthread под Windows!) другая концепция:
			\begin{itemize}
				\item Указатель на поток "--- сложный объект, который надо запрашивать у ОС и освобождать (как \t{FILE*}), а не просто переменная.
				\item Ресурсы потока освобождаются, когда он завершился и на него больше нет указателей.
				\item Нет разделения joinable/detached.
				\item Если кто-то может спросить состояние потока "--- у него есть указатель, значит, ресурсы потока ещё не освобождены.
			\end{itemize}
	\end{itemize}
\end{frame}
