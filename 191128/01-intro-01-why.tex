\section{Параллельные вычисления}
\subsection{Зачем}

\begin{frame}
	\tableofcontents[currentsection,currentsubsection]
\end{frame}

\begin{frame}{Скорость вычислений}
	\begin{itemize}
		\item Хочется обрабатывать всё б\'{о}льшие объёмы информации всё быстрее.
		\item Пример: эмуляция движений воздуха на планете Земля за ближайшие 48 часов (прогноз погоды).
		\item Пока работает эмпирический \href{https://ru.wikipedia.org/wiki/\%D0\%97\%D0\%B0\%D0\%BA\%D0\%BE\%D0\%BD\_\%D0\%9C\%D1\%83\%D1\%80\%D0\%B0}{закон Мура}: каждые два года плотность транзисторов удваивается.
		\item Раньше это означало увеличение частоты процессора в два раза.
		\item Уже нет: процессор с частотой 2.8 ГГц был представлен в 2004 году (Pentium 4 Prescott).
		\item С тех пор скорость работы повышалась, но другими способами: размер кэша, скорость памяти, периферии...
		\item Уже уткнулись в ограничения размера процессора из-за скорости света.
	\end{itemize}
\end{frame}

\begin{frame}[fragile]{Параллелизм}
	\begin{itemize}
		\item Иногда можно работать быстрее, не увеличивая частоту, распараллелив команды:
\begin{minted}{cpp}
int x = a * b * 10;  // Нужен блок умножения.
int y = a / b;       // Нужен блок деления.
\end{minted}
		\item Процессоры умеют это автоматически детектировать без участия программистов.
		\item Компиляторы умеют передвигать операции так, чтобы процессору было проще.
		\item В последние годы активно появляются многоядерные процессоры: впихнуть второе ядро оказалось проще оптимизации физических процессов.
		\item Также можно использовать мощь б\'{о}льшего числа компьютеров (\href{https://ru.wikipedia.org/wiki/Folding@home}{Folding@Home}).
	\end{itemize}
\end{frame}

\begin{frame}{На домашнем компьютере}
	Идеи распараллеливания полезны и где-то, кроме ускорения:
	\begin{itemize}
		\item
			Обычные задачи дома не требуют большой вычислительной мощи:
			\begin{itemize}
				\item Процесс обычно ждёт реакции пользователя, диска или сети.
				\item Вычисления длятся не больше нескольких секунд.
			\end{itemize}
		\item Хочется свободно переключаться между приложениями и слушать музыку в фоне.
		\item Если есть ресурсоёмкая задача, нестрашно, если она будет выполняться чуть медленнее.
		\item На телефоне одно ядро может целиком отрисовывать нетормозящий интерфейс, а другое "--- производить вычисления.
		\item Пример: AlarmUserHandler из предыдущего домашнего задания.
	\end{itemize}
\end{frame}
